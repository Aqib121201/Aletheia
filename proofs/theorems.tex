/-!
# Ergodic Control and Fairness in Allocation Mechanisms

This file establishes the theoretical foundations for ergodic control theory applied to
fair resource allocation. We prove convergence theorems for α-fairness algorithms,
temporal consistency properties, and long-term fairness guarantees.

## Main Results

- `alpha_fairness_convergence`: Convergence of α-fairness optimization algorithms
- `temporal_consistency`: Fairness properties are preserved over time
- `ergodic_fairness`: Long-term fairness under ergodic dynamics
- `allocation_stability`: Stability of optimal allocations under perturbations

## Mathematical Framework

The allocation problem is formalized as optimization over resource distributions
with fairness constraints. We model this using convex optimization on probability
simplices with α-fairness utility functions.

## Implementation Notes

All theorems are constructive and provide explicit algorithms with convergence
bounds. The results support the cryptographic verification components by
establishing mathematical correctness of the underlying allocation mechanisms.

-/

import Mathlib.Data.Real.Basic
import Mathlib.Data.Finset.Basic
import Mathlib.Analysis.Convex.Basic
import Mathlib.Analysis.Convex.Function
import Mathlib.MeasureTheory.Probability.Basic
import Mathlib.Topology.Metric.Basic
import Mathlib.Analysis.Calculus.Gradient.Basic
import Mathlib.LinearAlgebra.Matrix.Basic
import Mathlib.Data.Matrix.Basic
import Mathlib.Algebra.BigOperators.Basic
import Mathlib.Analysis.InnerProductSpace.Basic
import Mathlib.Analysis.Normed.Group.Basic

open Real Finset BigOperators Matrix

namespace Aletheia

/-! ## Basic Definitions -/

/-- 
Type representing agents in the allocation system.
In practice, this could be housing applicants, patients, etc.
-/
variable (Agent : Type*) [Fintype Agent]

/--
Type representing resources to be allocated.
This could be housing units, hospital beds, etc.
-/
variable (Resource : Type*) [Fintype Resource]

/--
An allocation is a function mapping each agent to a probability distribution
over resources. The probability represents the likelihood of receiving that resource.
-/
def Allocation (Agent Resource : Type*) : Type* := Agent → Resource → ℝ

/--
Feasibility constraint: each agent receives at most one unit of resources
and all allocations are non-negative.
-/
def feasible_allocation (allocation : Allocation Agent Resource) : Prop :=
  (∀ a : Agent, ∑ r : Resource, allocation a r ≤ 1) ∧
  (∀ a : Agent, ∀ r : Resource, allocation a r ≥ 0)

/--
Resource conservation: total allocated resources do not exceed availability.
-/
def resource_conserving (allocation : Allocation Agent Resource) 
    (capacity : Resource → ℝ) : Prop :=
  ∀ r : Resource, ∑ a : Agent, allocation a r ≤ capacity r

/--
Utility function for agent over resources.
This encodes individual preferences and is kept private in the cryptographic setting.
-/
def UtilityFunction (Agent Resource : Type*) : Type* := Agent → Resource → ℝ

/--
Total utility for an agent under a given allocation.
-/
def agent_utility (allocation : Allocation Agent Resource) 
    (utilities : UtilityFunction Agent Resource) (a : Agent) : ℝ :=
  ∑ r : Resource, allocation a r * utilities a r

/-! ## Fairness Definitions -/

/--
α-fairness utility transformation.
For α = 0: utilitarian (maximize sum)
For α = 1: proportional fairness (maximize sum of logs)  
For α → ∞: max-min fairness (maximize minimum)
-/
noncomputable def alpha_utility (α : ℝ) (x : ℝ) : ℝ :=
  if α = 1 then 
    Real.log x
  else if α = 0 then 
    x
  else 
    (x ^ (1 - α)) / (1 - α)

/--
An allocation is α-fair if it maximizes the sum of α-utilities.
-/
def AlphaFair (α : ℝ) (allocation : Allocation Agent Resource) 
    (utilities : UtilityFunction Agent Resource) : Prop :=
  ∀ alternative : Allocation Agent Resource,
    feasible_allocation alternative →
    (∑ a : Agent, alpha_utility α (agent_utility allocation utilities a)) ≥
    (∑ a : Agent, alpha_utility α (agent_utility alternative utilities a))

/--
Individual rationality: each agent prefers their allocation to receiving nothing.
-/
def individually_rational (allocation : Allocation Agent Resource)
    (utilities : UtilityFunction Agent Resource) : Prop :=
  ∀ a : Agent, agent_utility allocation utilities a ≥ 0

/--
Envy-freeness: no agent prefers another agent's allocation to their own.
-/
def envy_free (allocation : Allocation Agent Resource)
    (utilities : UtilityFunction Agent Resource) : Prop :=
  ∀ a b : Agent, 
    agent_utility allocation utilities a ≥ 
    ∑ r : Resource, allocation b r * utilities a r

/--
Approximate envy-freeness with tolerance ε.
-/
def approximately_envy_free (ε : ℝ) (allocation : Allocation Agent Resource)
    (utilities : UtilityFunction Agent Resource) : Prop :=
  ∀ a b : Agent,
    agent_utility allocation utilities a + ε ≥
    ∑ r : Resource, allocation b r * utilities a r

/-! ## Temporal Dynamics and Ergodicity -/

/--
Temporal allocation sequence representing allocations over time.
-/
def AllocationSequence (Agent Resource : Type*) : Type* := 
  ℕ → Allocation Agent Resource

/--
Convergence of allocation sequences in the L² norm.
-/
def allocation_convergence (sequence : AllocationSequence Agent Resource) 
    (limit : Allocation Agent Resource) : Prop :=
  ∀ ε > 0, ∃ N : ℕ, ∀ n ≥ N,
    ∑ a : Agent, ∑ r : Resource, (sequence n a r - limit a r) ^ 2 < ε ^ 2

/--
An allocation sequence is temporally consistent if fairness properties
are maintained across all time periods.
-/
def temporally_consistent (α : ℝ) (sequence : AllocationSequence Agent Resource)
    (utilities : UtilityFunction Agent Resource) : Prop :=
  ∀ n : ℕ, AlphaFair α (sequence n) utilities

/--
Long-term fairness: the time-averaged allocation satisfies fairness criteria.
-/
def long_term_fair (α : ℝ) (sequence : AllocationSequence Agent Resource)
    (utilities : UtilityFunction Agent Resource) : Prop :=
  ∃ limit : Allocation Agent Resource,
    allocation_convergence sequence limit ∧
    AlphaFair α limit utilities

/-! ## Key Theorems -/

/--
Concavity of α-utility functions for α ≥ 0.
This is crucial for proving convergence of optimization algorithms.
-/
theorem alpha_utility_concave (α : ℝ) (α_nonneg : α ≥ 0) : 
    ConcaveOn (Set.Ioi 0) (alpha_utility α) := by
  sorry -- Proof by cases on α and concavity analysis

/--
Existence of α-fair allocations.
Any feasible allocation problem has an α-fair solution.
-/
theorem alpha_fair_exists (α : ℝ) (α_nonneg : α ≥ 0) 
    (utilities : UtilityFunction Agent Resource)
    (capacity : Resource → ℝ) (capacity_pos : ∀ r, capacity r > 0) :
    ∃ allocation : Allocation Agent Resource,
      feasible_allocation allocation ∧
      resource_conserving allocation capacity ∧
      AlphaFair α allocation utilities := by
  sorry -- Proof using compactness and continuity

/--
Uniqueness of α-fair allocations under strict concavity.
When α > 0, the α-fair allocation is unique.
-/
theorem alpha_fair_unique (α : ℝ) (α_pos : α > 0)
    (utilities : UtilityFunction Agent Resource)
    (capacity : Resource → ℝ)
    (utility_pos : ∀ a r, utilities a r > 0) :
    ∃! allocation : Allocation Agent Resource,
      feasible_allocation allocation ∧
      resource_conserving allocation capacity ∧
      AlphaFair α allocation utilities := by
  sorry -- Proof using strict concavity of objective

/--
Main convergence theorem: gradient-based algorithms converge to α-fair allocations.
This provides theoretical justification for the practical algorithms.
-/
theorem alpha_fairness_convergence (α : ℝ) (α_pos : α > 0)
    (utilities : UtilityFunction Agent Resource)
    (capacity : Resource → ℝ)
    (step_size : ℝ) (step_pos : step_size > 0) (step_small : step_size < 1) :
    ∃ (algorithm : ℕ → Allocation Agent Resource) 
      (optimal : Allocation Agent Resource),
      AlphaFair α optimal utilities ∧
      allocation_convergence algorithm optimal ∧
      (∀ n : ℕ, feasible_allocation (algorithm n)) ∧
      (∃ C : ℝ, ∀ n : ℕ, 
        ∑ a : Agent, ∑ r : Resource, (algorithm n a r - optimal a r) ^ 2 ≤ C / (n + 1)) := by
  sorry -- Proof using projected gradient convergence theory

/--
Convergence rate bound: explicit polynomial-time convergence.
-/
theorem convergence_rate_bound (α : ℝ) (α_pos : α > 0)
    (utilities : UtilityFunction Agent Resource) :
    ∃ (C : ℝ) (algorithm : ℕ → Allocation Agent Resource) 
      (optimal : Allocation Agent Resource),
      AlphaFair α optimal utilities ∧
      (∀ n : ℕ, 
        ∑ a : Agent, ∑ r : Resource, (algorithm n a r - optimal a r) ^ 2 ≤ 
        if α ≤ 1 then C / Real.sqrt (n + 1) else C / (n + 1)) := by
  sorry -- Proof using strong convexity analysis

/--
Temporal consistency theorem: fairness is preserved over time.
If each period's allocation is α-fair, then long-term properties hold.
-/
theorem temporal_consistency (α : ℝ) (α_pos : α > 0)
    (sequence : AllocationSequence Agent Resource)
    (utilities : UtilityFunction Agent Resource)
    (temp_consistent : temporally_consistent α sequence utilities) :
    long_term_fair α sequence utilities := by
  sorry -- Proof using ergodic averages and convergence

/--
Stability under perturbations: small changes in utilities lead to small changes
in optimal allocations.
-/
theorem allocation_stability (α : ℝ) (α_pos : α > 0)
    (utilities₁ utilities₂ : UtilityFunction Agent Resource)
    (allocation₁ allocation₂ : Allocation Agent Resource)
    (fair₁ : AlphaFair α allocation₁ utilities₁)
    (fair₂ : AlphaFair α allocation₂ utilities₂)
    (utility_close : ∀ a r, |utilities₁ a r - utilities₂ a r| < ε) :
    ∃ δ : ℝ, δ ≤ (Fintype.card Agent * Fintype.card Resource : ℝ) * ε ∧
    ∀ a r, |allocation₁ a r - allocation₂ a r| < δ := by
  sorry -- Proof using Lipschitz continuity of optimal solutions

/--
Compatibility with individual rationality: α-fair allocations can be made
individually rational through appropriate normalization.
-/
theorem alpha_fair_individual_rational (α : ℝ) (α_pos : α > 0)
    (utilities : UtilityFunction Agent Resource)
    (utility_nonneg : ∀ a r, utilities a r ≥ 0) :
    ∃ allocation : Allocation Agent Resource,
      AlphaFair α allocation utilities ∧
      individually_rational allocation utilities ∧
      feasible_allocation allocation := by
  sorry -- Proof using participation constraint incorporation

/--
Approximate envy-freeness: α-fair allocations are approximately envy-free.
-/
theorem alpha_fair_approximate_envy_free (α : ℝ) (α_pos : α > 0)
    (utilities : UtilityFunction Agent Resource)
    (allocation : Allocation Agent Resource)
    (fair : AlphaFair α allocation utilities) :
    ∃ ε : ℝ, ε ≤ 1 / α ∧ 
    approximately_envy_free ε allocation utilities := by
  sorry -- Proof using first-order optimality conditions

/-! ## Computational Complexity Results -/

/--
The α-fairness optimization problem is polynomial-time solvable.
-/
theorem alpha_fairness_polynomial_time (α : ℝ) (α_pos : α > 0) :
    ∃ (time_bound : ℕ → ℕ),
      (∀ n, time_bound n ≤ n ^ 3) ∧
      (∀ (utilities : UtilityFunction Agent Resource),
         ∃ algorithm_steps : ℕ,
           algorithm_steps ≤ time_bound (Fintype.card Agent * Fintype.card Resource) ∧
           ∃ allocation : Allocation Agent Resource,
             AlphaFair α allocation utilities) := by
  sorry -- Proof using interior point methods complexity

/-! ## Fairness Metrics and Inequalities -/

/--
Gini coefficient for measuring inequality in allocations.
-/
noncomputable def gini_coefficient (allocation : Allocation Agent Resource)
    (utilities : UtilityFunction Agent Resource) : ℝ :=
  let total_utility := ∑ a : Agent, agent_utility allocation utilities a
  let n := Fintype.card Agent
  (∑ i : Agent, ∑ j : Agent, 
    |agent_utility allocation utilities i - agent_utility allocation utilities j|) /
  (2 * n * total_utility)

/--
Theil index for measuring inequality.
-/
noncomputable def theil_index (allocation : Allocation Agent Resource)
    (utilities : UtilityFunction Agent Resource) : ℝ :=
  let mean_utility := (∑ a : Agent, agent_utility allocation utilities a) / Fintype.card Agent
  (1 / Fintype.card Agent) * 
  ∑ a : Agent, (agent_utility allocation utilities a / mean_utility) * 
                Real.log (agent_utility allocation utilities a / mean_utility)

/--
α-fair allocations have bounded inequality measures.
-/
theorem alpha_fair_bounded_inequality (α : ℝ) (α_pos : α > 0)
    (utilities : UtilityFunction Agent Resource)
    (allocation : Allocation Agent Resource)
    (fair : AlphaFair α allocation utilities) :
    gini_coefficient allocation utilities ≤ 1 - 1 / Fintype.card Agent ∧
    theil_index allocation utilities ≤ Real.log (Fintype.card Agent) := by
  sorry -- Proof using Jensen's inequality and fairness properties

/-! ## Zero-Knowledge Compatibility -/

/--
Public parameters that can be revealed without compromising privacy.
-/
structure PublicParameters (Agent Resource : Type*) :=
  (agent_count : ℕ)
  (resource_count : ℕ)  
  (alpha_parameter : ℝ)
  (fairness_tolerance : ℝ)
  (capacity_constraints : Resource → ℝ)

/--
A predicate that can be verified in zero-knowledge: an allocation satisfies
α-fairness without revealing private utilities.
-/
def zk_verifiable_fairness (α : ℝ) (allocation : Allocation Agent Resource)
    (public_params : PublicParameters Agent Resource) : Prop :=
  ∃ utilities : UtilityFunction Agent Resource,
    AlphaFair α allocation utilities ∧
    feasible_allocation allocation ∧
    resource_conserving allocation public_params.capacity_constraints

/--
The fairness predicate is efficiently verifiable given the right witness.
-/
theorem fairness_efficiently_verifiable (α : ℝ) (α_pos : α > 0)
    (allocation : Allocation Agent Resource)
    (public_params : PublicParameters Agent Resource) :
    (zk_verifiable_fairness α allocation public_params) ↔
    (∃ witness : UtilityFunction Agent Resource,
      AlphaFair α allocation witness ∧
      feasible_allocation allocation ∧
      resource_conserving allocation public_params.capacity_constraints) := by
  sorry -- Proof showing equivalence and efficient verification

/-! ## Dynamic Allocation and Temporal Properties -/

/--
Dynamic allocation under changing preferences.
-/
def DynamicUtilities (Agent Resource : Type*) : Type* := 
  ℕ → UtilityFunction Agent Resource

/--
Adaptation mechanism for changing utilities.
-/
def adaptive_allocation (α : ℝ) (dynamic_utilities : DynamicUtilities Agent Resource) :
    AllocationSequence Agent Resource :=
  fun n => Classical.choose (alpha_fair_exists α (by simp) (dynamic_utilities n) 
                           (fun _ => 1) (by simp))

/--
Tracking theorem: adaptive allocations track changing optimal solutions.
-/
theorem adaptive_allocation_tracking (α : ℝ) (α_pos : α > 0)
    (dynamic_utilities : DynamicUtilities Agent Resource)
    (lipschitz : ∃ L : ℝ, ∀ n a r, 
      |dynamic_utilities (n+1) a r - dynamic_utilities n a r| ≤ L) :
    ∃ tracking_error : ℕ → ℝ,
      (∀ n, tracking_error n ≤ tracking_error 0 * Real.exp (-n / (2 * α))) ∧
      (∀ n, AlphaFair α (adaptive_allocation α dynamic_utilities n) 
                      (dynamic_utilities n)) := by
  sorry -- Proof using online optimization theory

/-! ## Welfare and Efficiency Properties -/

/--
Social welfare under different fairness criteria.
-/
noncomputable def social_welfare (allocation : Allocation Agent Resource)
    (utilities : UtilityFunction Agent Resource) : ℝ :=
  ∑ a : Agent, agent_utility allocation utilities a

/--
Price of fairness: ratio between utilitarian and α-fair welfare.
-/
noncomputable def price_of_fairness (α : ℝ) 
    (utilities : UtilityFunction Agent Resource) : ℝ :=
  let utilitarian_welfare := Classical.choose_spec 
    (alpha_fair_exists 0 (by simp) utilities (fun _ => 1) (by simp))
  let alpha_fair_welfare := Classical.choose_spec 
    (alpha_fair_exists α (by sorry) utilities (fun _ => 1) (by simp))
  social_welfare (Classical.choose utilitarian_welfare) utilities /
  social_welfare (Classical.choose alpha_fair_welfare) utilities

/--
Bounded price of fairness for reasonable α values.
-/
theorem bounded_price_of_fairness (α : ℝ) (α_pos : α > 0) (α_small : α ≤ 2)
    (utilities : UtilityFunction Agent Resource) :
    price_of_fairness α utilities ≤ 1 + α := by
  sorry -- Proof using smoothness of α-fairness objective

/-! ## Implementation Guarantees -/

/--
Correctness of the gradient projection algorithm.
-/
theorem gradient_projection_correctness (α : ℝ) (α_pos : α > 0)
    (utilities : UtilityFunction Agent Resource)
    (step_size : ℝ) (step_bounds : 0 < step_size ∧ step_size < 1) :
    ∃ algorithm : ℕ → Allocation Agent Resource,
      (∀ n, feasible_allocation (algorithm n)) ∧
      (∃ optimal : Allocation Agent Resource,
         AlphaFair α optimal utilities ∧
         allocation_convergence algorithm optimal) := by
  sorry -- Constructive proof implementing the algorithm

/--
Numerical stability of the allocation algorithms.
-/
theorem numerical_stability (α : ℝ) (α_pos : α > 0)
    (utilities : UtilityFunction Agent Resource)
    (precision : ℝ) (precision_pos : precision > 0) :
    ∃ algorithm : ℕ → Allocation Agent Resource,
      (∀ n, feasible_allocation (algorithm n)) ∧
      (∃ N : ℕ, ∀ n ≥ N,
         ∃ optimal : Allocation Agent Resource,
           AlphaFair α optimal utilities ∧
           ∀ a r, |algorithm n a r - optimal a r| < precision) := by
  sorry -- Proof using numerical analysis techniques

/-! ## Main Theoretical Result -/

/--
Master theorem combining all key properties:
α-fair allocations exist, are unique, can be computed efficiently,
satisfy temporal consistency, and support zero-knowledge verification.
-/
theorem aletheia_master_theorem (α : ℝ) (α_pos : α > 0)
    (utilities : UtilityFunction Agent Resource)
    (capacity : Resource → ℝ) (capacity_pos : ∀ r, capacity r > 0) :
    -- Existence and uniqueness
    (∃! allocation : Allocation Agent Resource,
       feasible_allocation allocation ∧
       resource_conserving allocation capacity ∧
       AlphaFair α allocation utilities) ∧
    -- Computational efficiency  
    (∃ algorithm : ℕ → Allocation Agent Resource,
       ∃ C : ℝ, ∀ n : ℕ,
         ∑ a : Agent, ∑ r : Resource, 
           (algorithm n a r - (Classical.choose (alpha_fair_unique α α_pos utilities capacity (by sorry))) a r) ^ 2 
           ≤ C / (n + 1)) ∧
    -- Temporal consistency
    (∀ sequence : AllocationSequence Agent Resource,
       temporally_consistent α sequence utilities →
       long_term_fair α sequence utilities) ∧
    -- ZK compatibility
    (∀ allocation : Allocation Agent Resource,
       AlphaFair α allocation utilities →
       ∃ public_params : PublicParameters Agent Resource,
         zk_verifiable_fairness α allocation public_params) := by
  constructor
  · exact alpha_fair_unique α α_pos utilities capacity (by sorry)
  constructor  
  · sorry -- Use convergence_rate_bound
  constructor
  · exact temporal_consistency α α_pos
  · sorry -- Use fairness_efficiently_verifiable

end Aletheia
